\documentclass{article}

\title{Artificial Intelligence assignment 2: Spam Filtering Using a Naive Bayes Text Classifier}
\author{Ilse Barf (S3500306), Galina Lesnic (SS3367398), Nathan van Beelen (S3392961)}

\begin{document}

\maketitle

\section*{3.3 Example runs}
\textit{What happens if you train and test on the same data?}

The accuracy will be higher. This is due to the fact that it can calculate
correlations that are apparent in this particular set of the data but do not
define the difference between regular and spam mails. In other words, you can
safely overfit on the train data and get a high accuracy score. Additionally,
we used the vocabulary of the train data, which will be exactly the same in the
test if we use the same data for testing. This is not the case if we use a
different data set for testing.

\section*{5 Final Questions}
\textit{The  data  used  in  this  assignment  contains  only  e-mails  in  the  English  language.
What happens if an e-mail in Dutch is given to your spam filter trained with English messages?
How  will  the  Dutch  message  be  classified?   Assume  that  there  are  no  common  words  in
English and Dutch.  Explain your answer.}
\begin{itemize}
    \item Nathan van Beelen: The messages will be classified according to the number of
          regular and spam messages used in the training data.
          This is because the classifier works by calculating the likelihood that a
          certain word is from a regular or spam mail. However, in Dutch it will not
          recognise any words. So this chance will be equal for both classifications.
          The only thing that differs is the chance that a mail is regular or spam based
          on the number of regular or spam mails. In the case this is also the same,
          the chances are equal. Due to the way we classify mails this means that
          mails will be classified as spam since the chance that it is regular needs
          to be bigger in order for it to be classified as regular.
    \item Ilse Barf:
    \item Galina Lesnic:
\end{itemize}

\textit{ The Naive Bayes assumption is that the attributes (or features) are independent.  Are the
words  in  a  message  really  independent?   And  what  can  you  say  about  the  independence
between and within bigrams?  Explain your answer in 250 words.}
\begin{itemize}
    \item Nathan van Beelen: No, the individual words are not independent. This is
                             because language has a certain structure. An adjective
                             needs a noun for example. Another example would be
                             semantics. The adjective 'new' would make sense in 
                             combination with the noun 'concept', but the adjective
                             'green' does not. This means that 'new' and 'concept'
                             have a semantic dependence, but 'green' and 'concept'
                             do not. Both in the case of between bigrams
                             and within bigrams there is a dependence. The dependence within
                             bigrams can be explained in a similar way as I did for the
                             individual words. Another example would be the determiner,
                             which will always be accompanied by either an adjective or
                             a noun on which it is dependent. In the case of the dependence
                             between bigrams it is mostly due to the dependence of words
                             to form a sentence (although the dependence can also be similar
                             to within bigrams if the words are split correctly).
                             To illustrate the dependence between bigrams, consider the
                             sentence 'Bob is playing soundly.' This will be split in:
                             'Bob is' and 'playing soundly.' The bigram 'Bob is' doesn't
                             make sense in and of itself. It needs a verb in order to make
                             sense. Therefore it is dependent on the bigram containg the
                             verb.
    \item Ilse Barf:
    \item Galina Lesnic:
\end{itemize}

\end{document}
